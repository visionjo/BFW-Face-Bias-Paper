\begingroup
\let\clearpage\relax 
\onecolumn 
\newpage
\newpage\newpage



\section*{TODO}
\subsection{Joe}
\begin{enumerate}
    \item  WRITING
    \begin{itemize}
        \item More complete lit review-- complete draft 1 of Related Works (1 week)
    \end{itemize} 
    \item DATA
    \begin{itemize}
        \item Confirm no overlap between Deomonpairs and training data + write about this/ resources in more depth in experimental setup & implementation (3/4 week)
        \item Verify data used to train arcface being used, put in report, along other with spec of network (footnote to its repo)
        \item Various figures showing daataset stats, like distribution of ethnicity, subjects per, faces per subject, etc., Make sure to produce in way updating is quick and easy, as plan to speak more about the goals for data. Figures and tables in report, with next versions looking and feeling (and sized) as intended for this draft/ at this point in time (2 days). In summary: Provide absolute data stats and format as table in paper. Update Fig 4 and Table 1. Discuss data goal (more on this under @Samson's TODO).
    \end{itemize}
    \item EXPERIMENTS
    \begin{itemize}
        \item Varying threshold
        \begin{itemize}
            \item Make DET for all pairs combines, and then see improvements with varying threshold (i.e., single list with all pairs, first run with single threshold found on held out set, and then another run with variable threshold as found to be best for each subgroup
        \end{itemize}
        \begin{itemize}
            \item Use ethnicity classifier to determine best threshold via LUT built on held-out validation data
            \item use NN list of subgroup to predict the subgroup in a semi-supervised manner (ie say use just the labels we have on our set and use on larger scale experiment with samples without attribute labels
        \end{itemize}
        \item FAR\@Rank 1 + 5 table for various thresholds and all (Need more empirical ways to evaluate and baseline) 
        \item Find evidence (interrupt model)— why Asian females fail— is there a common pattern in those that pass (checkout LIME)
    \item ANALYSIS
    \begin{itemize}
        \item How is the quality of images they fail against, i.e., low?
        \item  Alignment part— how do we do aligning Asian females? Face detection?
        \item Skin color of Asians in training data (i.e., check average face)

    \end{itemize}
    \end{itemize}
\end{enumerate}
\subsection{Samson}
\begin{enumerate}
    \item Sanity test with mxnet (Samson),  26 Friday 2019
    \item Patrick at NIST (Samson: Tomorrow set a time)
    \item Run repeat test with mx-net version of ArcFace (resnet 100)
    \item Reach out to IBM, for Diversity in Faces
    \item Work with Joe (and Gena if applicable, Nikil also if so) to determine goals for data, i.e., specifications of what we would like to include and means of obtaining more if needed.
    
\end{enumerate}
\subsection{Gena}
\begin{enumerate}
    \item Arrange to have quick powwow to get caught up with project, understand and help refine/improve goals and themes for this research track/ paper. Also, speak on what type of role you would like to play/ action items if any.
\end{enumerate}




\section*{IDEAS}
\noindent\texbf{(1) Fig. 1 - Improvements for Signal Detection Model}\\
-- make Fig 1 larger, magnify overlapping region and highlight as different color (i.e., maybe even using a simple clip art and dress it up with PP (or equivalent) to create attractive way to zoom... then, perhaps show sample pairs along the few, or, say, as (b) to the right of histograms, with one section for each graph--- fig : (a) visualization in Fig 1 with zoomed in effect |<horizontal dashed line>| (b) rows should pair samples (different type per row in some way highlighting where in the histogram each sample is found)

\noindent\texbf{(2) Fig. ~\ref{fig:detcurves}}\\
-- add lines depicting threshold for a specific rate (ie horizontal to put more emphasis on different in threshold).

-- Make 3 sperate PDFs and use latex subgraphs tighten up view, while adding subcaption references (i.e., (a) (b) and (c)


% \begin{figure}[h]
% \setlength{\unitlength}{0.14in} % selecting unit length
% \centering % used for centering Figure
% \begin{picture}(32,15) % picture environment with the size (dimensions)
%  % 32 length units wide, and 15 units high.
% \put(3,4){\framebox(6,3){$H_{B}(q)$}}
% \put(13,4){\framebox(6,3){$N[\cdot]$}}
% \put(23,4){\framebox(6,3){$H_{C}(q)$}}
% \put(0,5.5){\vector(1,0){3}}\put(9,5.5){\vector(1,0){4}}
% \put(19,5.5){\vector(1,0){4}}\put(29,5.5){\vector(1,0){3}}
% \put(-1,6.5) {$u(k)$}\put(30,6.5) {$y(k)$} \put(9.5,6.5)
% {$x_{B}(k)$}\put(19.5,6.5) {$x_{C}(k)$}
% \end{picture}
% \caption{An LNL Block Oriented Model Structure} % title of the Figure
% \label{fig:lnlblock} % label to refer figure in text
% \end{figure}

\section*{MY VIEW ON ADDING MORE AUTHORS AND COLLABORATIONS}
Some prefer rather fewer authors (surprisingly, more than I had imagined), but it is an important topic to be transparent about, so here are my feelings in the matter: I am 100\% for more authors if it means a better paper-- if there are ways to improve a paper, and someone has means to deliver on this, and then they deliver or are trusted to deliver, then offer them a part of this effort. This often inspires, which can directly translate to motivate. With that gives us an extra pair of hands to write, eyes and mind to revise, and added skills to enhance visuals (surely unique skill set should compliment deliverable, assuming the deliverable in itself is significant and is all expected of them (ie proposed model, building data collection pipeline and managing data collection + provide stats/ montages for database contributions, novel model idea and code, etc.,)). However, one noteworthy item and 110\% effort on improving paper and working on action items could always pay off provided good candidate. Thus, such a goal with promise as this paper has could inspire, motivate, and be appreciated by colleague. Thus, I am usually more about including then the opposite. Though, of course, they must contribute (which is something to speak about up front with complete transparency)... In summary: It is subjective, but the more the merrier 90\% of the time in my book (10\% only as a grad student when you need those 1 or 2 papers with just you and advisor, but that is not case of course + those are in the past at this point :) )>... that is my feelings on authorship in case you were wondering lol ;) Thus, if others you think would be interested please let me know and arrange to chat about it.... Also, notice ~80\% of the best papers have several, several authors, while the others are break-through ideas.

\endgroup
\twocolumn